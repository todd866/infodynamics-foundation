\documentclass[11pt]{letter}
\usepackage[margin=1in]{geometry}
\usepackage{hyperref}

\signature{Ian Todd\\Sydney Medical School\\University of Sydney}
\address{Sydney Medical School\\University of Sydney\\Sydney, NSW, Australia\\itod2305@uni.sydney.edu.au}

\begin{document}

\begin{letter}{Editorial Office\\IPI Letters}

\opening{Dear Editors,}

I am pleased to submit my manuscript, ``A Thermodynamic Foundation for the Second Law of Infodynamics,'' for consideration in IPI Letters.

This paper provides a thermodynamic mechanism for the second law of infodynamics proposed by Vopson and Lepadatu (2022). The core contribution is a bound showing that maintaining asymmetric, low-dimensional structure requires continuous work input. Without this energy expenditure, systems relax toward symmetric equilibrium---precisely the signature of the second law of infodynamics.

\textbf{Key insight:} I interpret Vopson's ``information entropy'' as \emph{structure-information}---the relative entropy $D_{\mathrm{KL}}(p \| p_{\mathrm{iso}})$ measuring a distribution's departure from isotropic equilibrium. This reconciles the two second laws: thermodynamic entropy increases in the bath while structure-information decreases in the system. No contradiction exists.

\textbf{Why this matters:} The second law of infodynamics has generated significant interest but invites the criticism that it contradicts thermodynamics. This paper resolves that tension using standard stochastic thermodynamics, showing that Vopson's law emerges naturally from the asymmetry between states that require maintenance and states that do not.

The manuscript includes:
\begin{itemize}
    \item A derivation of the geometric maintenance bound from Langevin dynamics
    \item Connection to housekeeping heat and maintenance power
    \item A minimal simulation demonstrating the relaxation signature
    \item Discussion of biological implications and Goldstone modes
\end{itemize}

The manuscript has not been published elsewhere and is not under consideration at any other journal. I note that the Editor-in-Chief is an author of the motivating works; I am happy for the handling editor or reviewers to be selected to avoid any conflict of interest.

\closing{Sincerely,}

\end{letter}
\end{document}
